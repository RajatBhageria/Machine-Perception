%\usepackage[T1]{fontenc}
\usepackage[latin9]{inputenc}
\usepackage[letterpaper]{geometry}
\geometry{verbose,tmargin=1in,bmargin=1in,lmargin=1in,rmargin=1in}
\usepackage{babel}
\usepackage{amsmath}
\usepackage{amssymb}
\usepackage{capt-of}
\usepackage{graphicx}
\usepackage[usenames,dvipsnames]{color}
\usepackage{latexsym}
\usepackage{xspace}
\usepackage{pdflscape}
\usepackage[hyphens]{url}
\usepackage[colorlinks]{hyperref}
\usepackage{enumerate}
\usepackage{ifthen}
\usepackage{float}
\usepackage{array}
\usepackage{tikz}
\usetikzlibrary{shapes}
\usepackage{algorithm2e}

%%%% HW instructions / collaboration text

\newcommand{\HWPolicies}{
\paragraph*{Instructions.} 

\paragraph*{Collaboration.} 
You are allowed and encouraged to work together. You may discuss the
homework to understand the problem and reach a solution in groups up
to size {\bf four students.} However, {\em each student must write
  down the solution independently, and without referring to written
  notes from the joint session. {\bf In addition, each student must
    write on the problem set the set of people with whom s/he
    collaborated.}} You must understand the solution well enough in
order to reconstruct it by yourself. (This is for your own benefit:
you have to take the exams alone.)
}

\newcommand{\ProgrammingPolicies}[1]{
\paragraph*{Instructions.}

\paragraph*{Collaboration.} 
For this programming assignment, you are {\bf not} allowed to
collaborate with other students. You may {\em discuss} the homework to
understand the problem, but {\bf you are not allowed to share your
code with any other students.} We will be using automatic checking
software to detect blatant copying of other student's assignments, so,
please, don't do it.
}

%%%% CUSTOM COMMANDS FOR FORMATTING EXAMS/HOMEWORKS
\newcounter{section_points}[section]
\newcounter{header_points}[section]
\newcounter{total_points}

\newcommand{\hpoints}[1]{
  \setcounter{header_points}{#1}
  \textbf{[#1 points]}
}

\newcommand{\points}[1]{
  \addtocounter{section_points}{#1}
  \addtocounter{total_points}{#1}
  \textbf{[#1 
    \ifthenelse{\equal{#1}{1}}
    {point}{points}]}
}

\newcommand{\point}{\textbf{[1 point]}}

\newboolean{ShowSolutions}
\newcommand{\Mistake}[2]{
  \ifthenelse{\boolean{ShowSolutions}}
  {\paragraph{\bf $\blacksquare$ COMMON MISTAKE #1:} {\sf #2} \bigskip}
  {}
}

\newcommand{\Solution}[2]{
  \ifthenelse{\boolean{ShowSolutions}}
    {
      \paragraph{\bf $\bigstar $ SOLUTION:} { \sf
        #1} \bigskip
    }
    { 
      #2
    } %} \vspace{1.5in}}
}
\newcommand{\out}[1]{}

\newboolean{ShowPointsInfo}

\newcommand{\PointStats}[0]{
  \ifthenelse{\boolean{ShowPointsInfo}}
  {
    \begin{center}

      \begin{tabular}{rl}
        \hline
        Stated Points: & \arabic{header_points} \\
        Section Points: & \arabic{section_points} \\
        Total Points So Far: & \arabic{total_points} \\
        \hline 
        \multicolumn{2}{c}{
          
          \ifthenelse{
            \equal{\value{section_points}}{\value{header_points}}
          }{CORRECT TOTAL}
          {{\bf INCORRECT TOTAL}}
          }
          \\
        \hline
      \end{tabular}
    \end{center}
  }{}
}

\newcounter{blankcount}
\newcommand{\myrepeat}[2] {
\setcounter{blankcount}{1}
\whiledo{\value{blankcount} < #1}{
#2
\addtocounter{blankcount}{1}
}
}

\newcommand{\blank}[1]{\underline{\myrepeat{#1}{\qquad}}}


%%%% CUSTOM MATH GOES HERE

\newcommand{\ind}[1]{\mathbf{1}\left(#1\right)}
\renewcommand{\Pr}{\mathbf{Pr}\xspace}
\newcommand{\Bern}{\textsf{Bernoulli}\xspace}
\newcommand{\sign}{\textsf{sign}}

\newcommand{\E}{\mathbf{E}}
\newcommand{\bx}{\mathbf{x}}
\newcommand{\bz}{\mathbf{z}}
\newcommand{\bw}{\mathbf{w}}
\newcommand{\bl}{\mathbf{\ell}}
\newcommand{\vc}[1]{\mathbf{#1}}

\newcommand{\Hypo}{\mathcal{H}}
\newcommand{\XX}{\mathcal{X}}
\newcommand{\cD}{\mathcal{D}}


% CIS580 packages
\usepackage{subfigure}
\usepackage{rotating}
\usepackage{verbatim}
\usepackage{graphicx}
% CIS580 notations
\newcommand{\R}{\mathbb{R}}
\newcommand{\Z}{\mathbb{Z}}
\newcommand{\C}{\mathbb{C}}

\usepackage{txfonts}
\newcommand{\ra}{\rightarrow}
\newcommand{\rect}{\text{rect}}
\newcommand{\F}{\mathcal{F}}
\newcommand{\w}{\omega}
\newcommand{\s}{\sigma}
\newcommand{\intinf}{\int_{-\infty}^{\infty}}
\newcommand{\suminf}{\sum_{n=-\infty}^{\infty}}
\newcommand{\fmap}{\multimapdotbothA}
\newcommand{\fmapvert}{\multimapdotbothAvert}

\newcommand{\ds}{\displaystyle}

% Vectors and matrices
\newcommand{\mat}[2]{
  \left[\begin{array}{#1} #2\end{array}\right]
}


% Dirac comb
\input cyracc.def
\font\tencyr=wncyr10
\def\cyr{\tencyr\cyracc}
\def\comb{\mbox{\cyr SH}}

\newcolumntype{M}{>{$\vcenter\bgroup\hbox\bgroup}c<{\egroup\egroup$}}


% COMMAND TO INSERT MATLAB LISTING
% Example
% \matlabscript{ocr/runHomework}{Script generating answers}

\usepackage{listings}
% This is the color used for MATLAB comments below
\definecolor{MyDarkGreen}{rgb}{0.0,0.4,0.0}

% For faster processing, load Matlab syntax for listings
\lstloadlanguages{Matlab}%
\lstset{language=Matlab,                        % Use MATLAB
        frame=single,                           % Single frame around code
        basicstyle=\scriptsize\ttfamily,             % Use small true type font
        keywordstyle=[1]\color{Blue}\bf,        % MATLAB functions bold and blue
        keywordstyle=[2]\color{Purple},         % MATLAB function arguments purple
        keywordstyle=[3]\color{Blue}\underbar,  % User functions underlined and blue
        identifierstyle=,                       % Nothing special about identifiers
                                                % Comments small dark green courier
        commentstyle=\usefont{T1}{pcr}{m}{sl}\color{MyDarkGreen}\scriptsize,
        stringstyle=\color{Purple},             % Strings are purple
        showstringspaces=false,                 % Don't put marks in string spaces
        tabsize=5,                              % 5 spaces per tab
        %
        %%% Put standard MATLAB functions not included in the default
        %%% language here
        morekeywords={xlim,ylim,var,alpha,factorial,poissrnd,normpdf,normcdf},
        %
        %%% Put MATLAB function parameters here
        morekeywords=[2]{on, off, interp},
        %
        %%% Put user defined functions here
        morekeywords=[3]{FindESS, homework_example},
        %
        morecomment=[l][\color{Blue}]{...},     % Line continuation (...) like blue comment
        numbers=left,                           % Line numbers on left
        firstnumber=1,                          % Line numbers start with line 1
        numberstyle=\scriptsize\color{Blue},    % Line numbers are blue
        stepnumber=5                            % Line numbers go in steps of 5
        }

\newcommand{\matlabscript}[2]
  {\begin{itemize}\item[]\lstinputlisting[caption=#2,label=#1]{#1.m}\end{itemize}}

